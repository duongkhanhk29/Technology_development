
Dear Editor,

I highly appreciate your constructive feedback on my research. My initial manuscript seems overly ambitious as it included many aspects of inequality, growth, and technology, which made the arguments insufficiently deep and coherent. Therefore, in the revised manuscript, I have removed the section on technology and inequality, focusing only on growth and inequality as a choice, as this part contains more interesting results and a stronger, more coherent argument.

Political choices that support the entire economy are referred to as ``economic purpose," while those that disproportionately favour a select group are termed ``bias purpose." One might debate the country and time effects in the models. The research objective is to determine the optimal values independently of constraints imposed by specific national contexts, including political institutions and inevitable laws influenced by factors such as time- or path-dependency. Thus, the goal is to isolate the effects of economic predictors from country- and time-related effects, rather than focusing solely on goodness-of-fit. I also experimented with panel dynamic models, but the optimisation results are fixed (one single output)—an outcome I was not seeking. Instead, I test for endogeneity to ensure that the residuals are uncorrelated with the predictors. 

I have revised nearly 90 per cent of the paper. My work is not only new in terms of concept (inequality as a choice) but also in its methodology (which is less common in economics but more frequently used in operational research). I have referenced Inequality: What Everyone Needs to Know (Oxford, 2016) and the augmented Kuznets Curve, as you suggested, and it has been extremely helpful in refining this research.


1. On technological change and inequality:  Created Unequal (Free Press, 1998). This book provides a severe critique of the skill-bias argument and an empirical study of the relationship between technological development (that is, the early rise of the tech sector) and inequality in the United States.  Several other papers on the UTIP site (see below) cover the effect of the sector on US inequality in later periods.

--> I have removed this part from the revised paper.


2. On inequality and development:  Inequality and Industrial Change (Cambridge, 2002). There is an extensive treatment of the Kuznets framework in this book, with a suggested explanation (the augmented Kuznets Curve) for the upward-sloping income-inequality relationship that you find (in rich countries). 

--> This book is really helpful to me, I have included quite a few arguments from it.

3. On inequality and economic instability:  Inequality and Instability (Oxford, 2012). 

--> I did not go too deep into this matter, but it also helped me understand economic instability.

4. Summary treatments:  Inequality: What Everyone Needs to Know (Oxford, 2016) and Entropy Economics (Chicago, 2025). 

Inequality: What Everyone Needs to Know (Oxford, 2016) is a good document, I referenced it to write about the origins of economic inequality in the paper.

Entropy Economics (Chicago, 2025) is a book that I am certain will be trending soon. Although the research content in the paper might not be highly related, I have indeed had several ideas about the entropy-approach for inequality research (such as measuring social mobility) and certainly, this book will be my reference.

5. Please be aware that the skill-bias hypothesis predates Acemoglu (2002) by about a decade.  And that DiNardo and Pischke had a biting critique already in 1997. It is by no means a consensus view despite the efforts of its proponents to present it as such.

--> Thanks for correcting me. Although the technology section has been removed, it helped me gain a more objective view of the skill-bias hypothesis.

6. Please also be aware that the SWIID, which you use, is heavily derived from the EHII dataset that I and my students developed, available at <http://utip.lbj.utexas.edu>.   Working paper 68, on the site, provides comparisons with other data sets. 

Further, the difference between market and disposable Ginis in the SWIID is highly problematic for two reasons.  First, "market" inequality measures are distorted in rich countries by the large number of households with zero market income, because they have adequate public pensions (and other non-market income) to live independently.  This, and not inherent "market" inequality, is the main reason why Solt's market inequality numbers (and the surveys on which they are based) for say Canada or Sweden are comparable to such countries as Brazil or Mexico, even though the societies they describe are vastly more egalitarian.  Second, Solt derives his disposable income measures for many countries by a simple imputation, assuming that the differences from his market measure are similar across countries. The gap, which you use as a measure of redistribution, is therefore not an independent measurement in many, if not most, cases.  Again see the appendices in UTIP working paper 68 for many actual measures, of "market," gross and disposable income inequalities, along with Solt's estimates. 

--> I agree with this opinion. Therefore, I used LIS to create findings, but due to limited coverage, I performed a robust check with SWIID.

7. Please also be aware that Branko Milanovic has updated the "Elephant Curve" analysis and reports that it no longer provides a valid description of recent developments. 
--> Thanks, I noted.




 