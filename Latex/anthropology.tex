\subsection{Anthropologically, unequal societies prevail over equal ones}

The study of the development of human society has shown that the earliest social forms, those of hunter-gatherers, were the most egalitarian of all the forms of human society that have existed to this day. An immediate economy with restricted savings and accumulation, which in turn limits commercial activity and economic interdependence, is the means by which equality can be attained \parencite{woodburn1982egalitarian}. One of the central hypotheses for the emergence of egalitarianism from such societies is the implementation of egalitarian cultural norms \parencite{boehm2009hierarchy}. Nonetheless, the non-competitive value systems of hunter-gatherer societies limited agricultural development due to sharing rules that restricted investments and savings during periods of rapid population growth and high demand for food. This often resulted in resource depletion, competition, and conflict. However, when agricultural production levels were raised and a surplus was created, there was a greater demand for specialized fields like management and trade, leading to the gradual emergence of social classes \parencite{carneiro1970theory}.

Simple hunter-gatherer societies with limited food resources, low population densities, and a nomadic lifestyle tend to be less hierarchical with weak property rights, a strong sharing ethic, and little economic competition. On the other hand, hierarchical societies have superior resources and higher population densities, engage in production competition, and convert surpluses into commercial commodities, leading to more pronounced economic competition \parencite{hayden2001richman}. Although economic competition has been shown to be a manifestation of unequal societies, scientific evidence demonstrates that inequalities can persist in cases of economic cooperation in the presence of specialization and trade. This is the result of cultural learning, which occurs when people tend to learn from those who are more economically successful, thereby disseminating cultural knowledge that promotes economic success. These processes can generate a stable, culturally transmissible division of labour. This type of social stratification is preferred because it generates more surplus value and distributes it more equitably when there is greater specialization and economic success is highly regarded by society \parencite{henrich2008division}. In addition, stratified societies have a competitive advantage in adapting to natural resources, which justifies their dominance over egalitarian societies. In constant environments, unequal access to resources can destabilize demographics, cause outward migration, and promote the proliferation of unequal societies. By sequestering mortality in the lower classes, stratified societies are better able to cope with resource shortages in changing environments \parencite{rogers2011spread}. Therefore, an unequal society always has an advantage over an equal society, regardless of the circumstances.


\ref{tab:country}