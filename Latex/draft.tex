\section{state of nature}

\begin{figure}[H]
    \centering
    \scalebox{1}{\includesvg{figs/lorenz.svg}}
    \caption{Lorenz curve}
    \captionsetup{font=footnotesize}
    \caption*{\textbf{Notes:} }
\end{figure}


All human societies have inequality levels between the two extremes of perfect equality (or Gini = 0) and extreme inequality (or Gini = 1), as measured by the Lorenz Curve; and it is impossible for a society to move toward perfect equality (refer to the feasible equality levels in countries estimated by \textcite{park2021getting} for more information). Inequality is frequently associated with a negative connotation and achieving (perfect) equality is impossible; thus, every society is evaluated by a negative metric rather than an objective measure of income distribution. Instead, a subjective inequality index, such as the Atkinson index as a function of social welfare based on the development of the inequality aversion parameter \parencite{atkinson1970measurement}, may be more appropriate. \textcite{lambert2003inequality} assert, based on mathematical analysis, that there exists a 'natural rate' of subjective inequality when all societies perceive and tolerate the same level of subjective inequality among their citizens. The findings of this study fully support the fair inequality hypothesis under conditions where society accepts a certain level of inequality.

Applying the principle of fair division, \textcite{park2021getting} figure out a feasible inequality (hereafter called optimal inequality) in countries based on the principle of maximizing total social welfare. Accordingly, fair income distribution is calculated using the Boltzmann distribution, an entropy-based method – comparable to Atkinson's \parencite*{atkinson1970measurement} method for calculating the subjective inequality index. Human emotions, irrational decisions, and deceptive behaviour are all limited by the Boltzmann fair distribution based on the principle of maximum entropy (or social welfare), and the Boltzmann probability entirely determines the participants’ economic share \parencite{park2022boltzmann}. The free-market societies (much like the Game world without state intervention) is equal to Gini index = 0.65 as calculated by \parencite{fuchs2014behavioral}. Intriguingly, this value is close to the Gini index of the variance of the Gaussian distribution, which equals 0.636.


\textcite{berg2017inequality} argue that `over longer horizons, avoiding excessive inequality and sustaining economic growth may be two sides of the same coin.' This suggests that reducing excessive inequality can foster more sustainable economic growth over time. However, it is important to note that excessive inequality differs significantly from moderate controlled increases in inequality, and sustainable growth is not the same as short-term or general economic growth. Therefore, this statement cannot be simplistically interpreted as implying that greater equality and greater growth always go hand in hand. This complex interplay has been examined through various theoretical and empirical lenses. The Kuznets curve posits that economic development initially increases inequality due to concentrated savings and industrialisation, but this effect diminishes over time with state intervention, demographic changes, and technological progress \parencite{kuznets1955economic}. Meanwhile, \textcite{milanovic2016global} argues that inequality follows a sinusoidal pattern in the long term, shaped by technological progress and globalisation, rather than the inverted U-shape suggested by Kuznets. Similarly, \textcite{piketty2014capital} contends that without effective capital taxation, inequality will persist or worsen, contradicting Kuznets's expectation of declining inequality in advanced economies. Empirical findings are mixed \parencite{martinez2020inequality}, with some studies supporting the Kuznets hypothesis and others finding no consistent link between growth and inequality.

Regardless, do we have a choice in inequality? According to Piketty’s argument, inequality can be considered inevitable, as he posits that when the return on capital (r) exceeds economic growth (g), without strong wealth redistribution, wealth concentration is perpetuated, making inequality a fundamental characteristic of unregulated capitalism \parencite{piketty2014capital}. Additionally, natural and social laws that transcend political control also play a role. For example, \textcite{scheffer2017inequality} show that the distribution of wealth in human societies parallels the abundance of species in the Amazon Forest, a natural system unaffected by human influence. Their research suggests that natural laws may partly explain societal inequality. Furthermore, \textcite{fuchs2014behavioral} find that inequality in the virtual economy of the Pardus game mirrors patterns in real-world economies such as Sweden and the UK, indicating that factors beyond political mechanisms contribute to the persistence of inequality. However, most economists argue that inequality is a choice. As Nobel Prize-winning economist Joseph Stiglitz states\footnote{Ford Foundation. (n.d.). \textit{Joseph Stiglitz on inequality and economic growth}. Retrieved November 19, 2024, from \href{https://www.fordfoundation.org/news-and-stories/big-ideas/inequalityis/joseph-stiglitz-on-inequality-and-economic-growth/}{https://www.fordfoundation.org/news-and-stories/big-ideas/inequalityis/joseph-stiglitz-on-inequality-and-economic-growth/} }, `we’ve chosen, in effect, to create a society with this great divide between the rich and the poor.' Namibia provides a counterexample to the view that inequality is inevitable \parencite{lawson2017inequality}. The country has successfully reduced inequality through deliberate policy interventions, such as land reform, progressive taxation, and social welfare programmes. These efforts have helped reduce wealth disparities, making Namibia a notable case where inequality has been mitigated through targeted state policies.

After all, inequality is ultimately determined by redistribution, a political choice (as seen in the case of Namibia), although initial inequality may be beyond human control (as evidenced in natural systems and virtual game worlds). If inequality is indeed a choice, do we prefer reduced inequality over economic growth? This leads us back to the question of whether there is a trade-off between inequality and growth. Although most of us agree that excessive inequality cannot lead to sustainable growth, the question remains whether there is a trade-off between growth and inequality when inequality is at controlled levels. Here, we return to the distinction between excessive and moderate inequality. Moderate inequality may not necessarily be harmful in terms of fairness \parencite{ku2013procedural} and could even be beneficial to growth, for example, by fostering competition \parencite{ferreira2022impact}. 

This study seeks to answer the question: if we set aside all non-economic factors, would we still prioritise reducing inequality over boosting growth, with the understanding that excessive inequality hinders future growth? This approach sheds light on the nature of economic development, distinct from national development, which includes both economic and non-economic factors. To address this question, it is essential to consider a broad range of economic predictors that capture the complex relationship between inequality and economic performance. These predictors help us understand the trade-offs between achieving higher economic growth (measured by GDP per capita) and reducing income inequality (measured by the Gini index), which are central to this inquiry.




\textbf{Power laws:} When income follows a power-law distribution at the upper end, inequality can be particularly pronounced, resulting in a greater number of ultra-rich individuals than would be expected under a log-normal distribution. This phenomenon, often referred to as a “fat tail,” reflects the mathematical properties of power laws. The power-law distribution also underpins the well-known 80–20 rule, originally identified by Vilfredo Pareto, who observed that 20\% of Italians owned 80\% of the land. Since then, numerous variations of this principle have been documented, suggesting that power-law distributions are prevalent across various real-world domains.






However, if development is a choice, we have chosen to prioritise growth over reducing inequality as we have transitioned from the equal, yet poor, societies of hunter-gatherers to the wealthy but unequal modern societies marked by discrimination, unrest, and war. As the second moment of income distribution, income inequality becomes a secondary concern after the first—mean income or GDP per capita. 