This study provides evidence that when economic development—encompassing both growth and inequality—is treated as an objective that benefits the entire economy rather than specific agents, growth and inequality can coexist and have a positive relationship. While some may be concerned that such a trade-off inevitably leads to extreme inequality that could hinder future growth, the findings indicate that this trade-off occurs within controlled bounds of inequality, as measured by a Gini index below 28. Although extending growth might suggest an increase in inequality, practical constraints and the inherent limits to growth prevent such extremes. The analysis utilises multi-objective optimisation, conceptualising both inequality and growth as policy objectives. The model identifies optimal strategies that maximise growth and minimise inequality within the observed range of GDP per capita, with deviations from these optimal choices interpreted as penalties reflecting non-economic factors. The results further support the hypothesis of a natural rate of subjective inequality.