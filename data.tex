\begin{table}[h!]
\centering
\footnotesize
\renewcommand{\arraystretch}{1.5} % Increase row height
\caption{Variables, Descriptions, and Data Sources}
\label{tab:var}
\resizebox{\textwidth}{!}{%
\begin{tabular}{|p{1.5cm}|p{2.2cm}|p{7cm}|l|}
\hline
\textbf{Type} & \textbf{Variables} & \textbf{Description} & \textbf{Data Source} \\
\hline
\textbf{Outcome} & GDP per capita & GDP per capita in constant 2015 US\$. & World Bank \\
\cline{2-4}
& Gini Index & Gini index of disposable income, measuring income inequality. & LIS/SWIID\\
\hline
\textbf{Predictors} & Redistribution & Difference between market income Gini and disposable income Gini, reflecting the impact of redistributive policies.& LIS/SWIID\\
\cline{2-4}
& Human Capital & Educational attainment and quality. & PWT \\
\cline{2-4}
& Technology & Total factor productivity, capturing innovation and skills. & PWT \\
\cline{2-4}
& Trade & Trade as a percentage of GDP. & World Bank \\
\cline{2-4}
& FDI & FDI net inflows as a percentage of GDP. & World Bank \\
\cline{2-4}
& Credit & Domestic credit to the private sector as a percentage of GDP. & World Bank \\
\cline{2-4}
& Fertility & Total births per woman (fertility rate). & World Bank \\
\cline{2-4}
& Unemployment & Unemployment rate as a percentage of the total labour force (using ILO modelled estimates). & World Bank \\
\cline{2-4}
& Inflation & Annual percentage change in consumer prices. & World Bank \\
\hline
\end{tabular}
}
\vspace{0.2cm}
\captionsetup{font=footnotesize}
    \caption*{\textbf{Notes:} LIS stands for Luxembourg Income Study, SWIID stands for Standardised World Income Inequality Database, PWT refers to the Penn World Table, FDI means Foreign Direct Investment, GDP is Gross Domestic Product, and ILO stands for the International Labour Organization. The data has been extracted and is current as of November 2024, reflecting the most recent updates available.}
\end{table}